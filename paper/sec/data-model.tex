\section{Data model}
\label{sec:data-model}

The criminological domain handles a wide variety of entities and relations between them.
When representing a criminal network with a graph theory model, this variety must be addressed.
Nodes and edges should not be limited to represent people and their mutual knowledge, but also places, assets, not yet investigated groups and every kind of interaction between them. 

In this scenario, a generic node should be labeled with an IRI pointing to an external knowledge base.
A generic edge should be undirected and weighted according to the EWMA  relevance of the observed interaction. 

Furthermore, also the investigation dynamics should be considered. Since entities are observed in their interactions, isolated nodes has no sense in this model. This means that a node insertion occurs only through the insertion of an interaction in which it is involved.

Such a general model makes the link mining task more flexible and adaptable to different criminal contexts.