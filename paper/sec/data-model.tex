\section{Data model}
\label{sec:data-model}

A \textit{criminal network} is a social structure made up of related actors sharing some criminal intent \cite{von2001organisierte}. The criminological domain deals with a wide variety of entities and relations between them, collected by heterogeneous data sources. When representing a criminal network with a graph theory model, this variety must be addressed.

We model such a network with the weighted undirected graph $G(V,E,w)$, where $v\in V$ represents a person, a group, an assets or a place, and $E$ represents their relations, for which the tipology and relevance are encapsulated by the correspondent weight.
A generic node should be labeled with an IRI pointing to an external knowledge base.
A generic edge should be undirected and weighted according to the Exponential Weighted Moving Average (EWMA) relevance of the observed interaction.  Such a general model makes the link mining more flexible and adaptable to different criminal contexts and investigative practices.

Furthermore, also the investigation dynamics should be considered. Since entities are observed in their interactions, isolated nodes has no sense in this model. This means that a node insertion occurs only through the insertion of an interaction involving it.

