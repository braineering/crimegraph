\section{Background}
\label{sec:background}
We assume that the reader is familiar with the standard graph terminology. However, we make a brief recap to introduce what we will see in the next sections.

Let $G(V,E)$ an undirected graph, where $V$ is the the set of vertices and $E$ is the set of edges, we representing the topological structure of a network, in which each node $v \in V$ is an individual or an organization, and each edge $e=(u,v)\in E$ represents a link (i.e. an interaction) between two nodes, $u$ and $v$. 

A \textit{path} $\pi_{x,y}$ between two nodes $x$,$y$ of \textit{length} $k$ in $G$ is a sequence of vertices $v_{1},v_{2},...,v_{k}$, where $x = v_{1}$ and $y = v_{k}$, such that the edge $(v_{i},v_{i+1}) \in E$ for $i = 1, 2,..., K-1$. The distance between $x$ and $y$, $d(x,y)$ is the length (number of edges) of the shortest path from $x$ to $y$ (or viceversa).

An undirected graph is \textit{connected} if, for every pair of nodes, there is a path between them. Otherwise, if a graph is not connected, it's constituted by at least two \textit{connected components}.

Given two nodes $x,y \in G$, we denote with $\Gamma(x)$ the set of neighbors of $x$ (i.e. adjacent nodes) and with $\Gamma_{x,y} = \Gamma(x) \bigcap \Gamma(y)$ the set of common neighbors between $x$ and $y$.

In this paper, we lay our focus on a criminal scenario and represent this with a graph on which to apply the link mining techniques. In particular, we wont predict the likelihood that a link, between two entities, is hidden or which may arise in the future. These problems are called respectively the \textit{link detection problem} and \textit{link prediction problem}. Briefly, we refer to the first case when the link between two nodes it's may already exist, but not visible, while we refer to the second case to indicate the prediction of future interaction between two observed individuals\cite{Hasan2011}.

Formally, for two times $t$ and $t' > t$, we denote $G[t,t']$ the subgraph of G consisting of all edges with a timestamp between $t$ and $t'$. Identified four times: $t_{0}, t'_{0}, t_{1}, t''_{1}$, with $t_{0}<t'_{0}<t_{1}<t'_{1}$, we refer to $[t_{0},t'_{0}]$ as the \textit{training interval} and $[t_{1},t'_{1}]$ as the \textit{test interval}. Applying a \textit{prediction algorithm} on the graph obtained after the training interval, as $G[t_{0},t'_{0}]$, we want to make a prediction on the edges that will be present in the graph $G[t_{1},t'_{1}]$ and not present in $G[t_{0},t'_{0}]$\cite{Liben-Nowell}.

In this work, the notions of training interval and test interval, are provided with the only purpose to test the algorithm that will be presented in the following. After all, this application produce an output of links mining in real time: in which each instant $t\textsuperscript{*}>t_{0}$ identifies a graph $G[t_{0},t\textsuperscript{*}]$ on which to execute the algorithm. More details of a real time processing are explained in Section \ref{sec:architecture}.

Currently, the problem of link prediction, as the link detection, is a long-standing challenge in modern information science. There are many algorithms for this problem and ways to address this challenge. The mainstreaming class of algorithms so-called \textit{similarity-based algorithms}. Other algorithms are based on \textit{Markov chains} or on \textit{statistical models}, as extensively discussed in \cite{Liben-Nowell} and \cite{Lu2011}.

Let us consider the link prediction, but the link detection is analogous because the metrics used for this task are the same adopted: the difference concerns how results are interpreted. The similarity metrics are so many and differ for the size with which focus the analysis. In detail, let $P_{x,y}$ the likelihood (i.e. the \textit{score}) that between the nodes $x$,$y$ there will be a link in the future, provided by the algorithm that implements the similarity considered metric, and let $\pi_{x,y}$ a path between them, we denote \textit{local indices} the metrics that can only compute the $P_{x,y}$ at distance $d_{\pi}(x,y) = 2$ (i.e it's applicable on only nodes that have a common neighbor $\in \Gamma_{x,y}$). Other types of metrics are \textit{quasi-local} indices, that consider $d(x,y) \geq i$, where $i$ identifies a neighborhood with min length equal to $i$ steps, starting from the node $x$ to $y$. Another metrics are \textit{global indices}, that computes the score between any pairs of nodes in $G$ not directly connected\cite{Lu2011}. The metrics used in this work are presented in Section \ref{sec:link-detection} and \ref{sec:link-prediction}.

PARLARE DELLE METRICHE DI PRECISIONE PER TESTARE LE METRICHE DI LINK PREDICTION

Notice that, in this work we assume that the graph is connected, therefore each pair of nodes is connected by a path. If we had in the case of a not connected graph, with a large connected component, as \textit{giant component}, and other small size connected components, the accuracy is distorted by different \textit{density} of edges in a connected components.

%Link mining refers to data mining techniques that consider data links when building descriptive or predictive models \cite{getoor2005link}.
