\section{Background}
\label{sec:background}

% THEORETICAL BACKGROUND

In this section we briefly introduce the notation and concepts to support the formal definition of our algorithms and metrics.

Let $G(V,E)$ be an undirected weighted graph, where $V$ is the set of undirected vertices, $E$ is the set of edges and $w:E\rightarrow\mathbb{R}$ is the weight function. 
%
A \textit{path} $\pi_{x,y}$ between two vertices $x$,$y$ of \textit{length} $k$ in $G$ is a sequence of vertices $v_{1},v_{2},\ldots,v_{k-1},v_{k}$, where $x = v_{1}$ and $y = v_{k}$, such that the edge $(v_{i},v_{i+1}) \in E$ for $i = 1, 2,\ldots,k-1$. 
%
Given an edge $(x,y)$, the distance $d(x,y)$ between $x$ and $y$, is the length of the shortest path from $x$ to $y$; notice that $d(x,y)=d(y,x)$.
%
A graph is \textit{connected} if there is a path between every pair of vertices. 
Otherwise, the graph contains at least two \textit{connected components}, i.e. connected subgraphs.
%
Notice that, in this work we assume a graph made up of one connected component only, namely a \textit{giant component}.

Given two vertices $x,y \in V$, we denote with $\Gamma(x)$ the set of neighbors of $x$ and with $\Gamma(x,y) = \Gamma(x) \bigcap \Gamma(y)$ the set of common neighbors between $x$ and $y$. Given a vertex $x \in V$, we denote with $k_{x}$ the degree of $x$.

The \textit{link mining} refers to data mining techniques that consider links as first-class citizens when defining descriptive and predictive models~\cite{getoor2005link}. It comprises community detection, entity classification, link detection, link prediction and more. 

%In this work we focus on link detection and prediction, namely the task of determining existent though invisible links, and the prediction of future interactions\cite{Hasan2011}.
Currently, the mainly adopted approaches for link mining are based on \textit{structural metrics} \cite{berlusconi2016link,Liben-Nowell,Lu2011}.
The structural approach classifies links according to metric scores that encapsulate relevant topological features. 
These metrics can be \textit{local}, \textit{quasi-local} and \textit{global}, depending on the locality of the topological features examined with respect to the considered entities. 
When tuning the locality degree, for a specific network, the  tradeoff between accuracy and computational complexity must be addressed.
%Quasi-local metrics are a good tradeoff of accuracy and computational complexity, because the global indices require considerable computational effort, while local indexes are limited. 

Link detection and prediction tasks are based on the same metrics, mainly differing in the setting up of the testing framework and the interpretation of experimental results.

%Let us consider the link prediction, because the link detection is analogous, the metrics used for this task are the same adopted: the difference concerns how the results are interpreted. The similarity metrics are so many and they and differ because the size in which focus the analysis. 
%In detail, let $s_{x,y}$ the likelihood, i.e. the \textit{score}, as \textit{proximity} or \textit{similarity}, that between the nodes $x$,$y$ there will be a link in the future, provided by the algorithm that implements the similarity considered metric, and let $\pi_{x,y}$ a path between them, we denote \textit{local indices} the metrics that can only compute the $s_{x,y}$ at distance $d_{\pi}(x,y) = 2$ (i.e it's applicable on only nodes that have a common neighbor $\in \Gamma(x,y)$).

%Other types of metrics are \textit{quasi-local} indices, that consider $d(x,y) \geq t$, where $t\in \mathbb{Z}, t\geq 1 $, as \textit{locality degree}, identifies a neighborhood with a max length equal to $t$ steps, starting from the node $x$ to $y$. Quasi-local metrics are a good tradeoff of accuracy and computational complexity, because the global indices require considerable computational effort, while local indexes are limited. 

%Other types of metrics are \textit{quasi-local} indices, that consider $d(x,y) \geq t$, where $t\in \mathbb{Z}, t\geq 1 $, as \textit{locality degree}, identifies a neighborhood with a max length equal to $t$ steps, starting from the node $x$ to $y$.
%Another metrics are \textit{global indices}, that computes the score between any pairs of nodes in $G$ not directly connected\cite{Lu2011}. The metrics used in this work are presented in Section~\ref{sec:link-detection,sec:link-prediction}.

% leverei la descrizione di RA e LP, lasciando solamente la citazione dell'articolo che ne parla.
%, that we report for simplicity:
%\begin{equation}
%\label{eqn:resource-allocation}
%s_{xy}\textsuperscript{RA} = 
%\sum\limits_{z\in\Gamma(x,y)}\frac{1}{k_{z}}
%\end{equation}
%where $k_{z}$ is the degree of node $z$.
%
%\begin{equation}
%\label{eqn:local-path}
%s_{xy}\textsuperscript{LP(k)} = 
%A^{2} + \epsilon A^{3} + \epsilon^{2}A^{4}+...+\epsilon^{k-2}A^{k}
%\end{equation}
%where $k\geq 1$ is a locality degree, $A$ is the adjacency matrix, and $\epsilon$ is a free parameter. For example, $A_{xy}^{2}$ is also the number of different paths with length $2$ connecting $x$ and $y$.

%Both metrics for link prediction assigned to a pair $x$,$y$ a score, representing their possible future relationship. This score can be viewed as a measure of similarity between nodes $x$ and $y$. Notice that for each pair of nodes, $x$, $y \in V$, we have that $s_{xy} = s_{yx}$. Performing an ranking among the RA score on a graph $G$, can be get the top most likely link those still not present in G.

% DATA MODEL

A \textit{criminal network} represents the social structure of people that are involved in some criminal intent and the related interactions~\cite{von2001organisierte}.
%
The criminological domain deals with a wide variety of entities and relations, collected by heterogeneous data sources. For example, relations between people and places could be tracked with malewares and IoT sensors. 
When analyzing criminal networks leveraging SNA models, this variety must be addressed \cite{pramanik2016framework}.

We model such a network with the weighted undirected graph $G(V,E)$, where  $ V$ represents the 
 of \textit{criminal nodes} (e.g. peoples, groups or places), and $E$ represents their \textit{criminal relations}, whose type and relevance are encapsulated by the correspondent weight function $w:E\rightarrow~\mathbb{R}_{\geq1}^{+}$. 

A generic node is labeled with an \textit{Internationalized Resource Identifier} (IRI), thus giving the possibility to exploit the power of inference. 
%\hlb{rivedere: points, so as to then exploit the power of inference}.
A generic undirected edge is weighted according to the \textit{Exponential Weighted Moving Average} (EWMA) applied to relevance of the observed interaction. 
Such a model distributes relative relevance of the recent observations with respect to past ones \cite{lucas1990exponentially}. 
For example, a wiretap records datasets would induce a weight of edge $x,y$ proportional to the call length between $x$ and $y$. 
Such a general model makes the link mining tasks more flexible and adaptable to different criminal contexts and investigative practices.

% LINK DETECTION

The link detection is the task of discovering \textit{hidden links} in a network. 
A link $(x,y)$ is hidden if it exists in reality, but it is missing within the available datasets, because it has been neglected during investigations or it has been concealed by criminals.
%
% LINK PREDICTION
%
The link prediction is about inferring the interactions that are likely to be created within the graph in the near future. 
%Formally, given a graph $G(V,E,w)$, we define a view of the graph at time $ t $ as its snapshot in $ t $, $G(V^t,E^t,w^t)$, which comprises the set of verticies $ V^t $, edges $ E^t $, and the weight function $ w^t(\cdot) $ as observed in $ t $.
%
%Considering two points in time $t$ and $t'$, with $t'>t$, and the resulting view of the same graph $G(V^t,E^t,w^t)$ and $G(V^{t'},E^{t'},w^{t'})$, we define as \textit{potential link} a link $(x,y)$ that does not exist at time $ t $ \textit{and} exists at time $ t' $, i.e.,  $(x,y) \notin E^t \wedge (x,y) \in E^{t'} $. 


