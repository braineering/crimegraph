\section{Related Works}
\label{sec:related-works}

The adoption of SNA in criminology began in the 90s \cite{berlusconi2017social}, providing evidence of positive experience as investigative tool in recent years \cite{van2009introduction}. 
%
Since then, SNA models and techniques have been strongly improved.
%
Nevertheless, SNA solutions are designed for a batch processing approach and cannot be easily adapted for the execution in a DSP application.
%
Moreover, although SNA metrics have been widely experimented in the criminology context~\cite{berlusconi2016link}, only recent works focused on their development as a Big Data analytics solution~\cite{pramanik2016framework}. Such a solution gives the possibility to extract in real-time valuable information from evolving dataset. 

In this context, it is necessary to develop new SNA models that are aware of criminological assumptions and that are ready to work with data streams~\cite{xu2005criminal,xu2004analyzing}.
%
The traditionally adopted metrics presented in Section~\ref{sec:classical-metrics} are mainly focused on vertices degree and number of common neighbors as the only index of similarity between vertices.
%
On the other hand, our new metrics take into account also the distribution of the weight of edges, looking for evidence of criminological assumptions.
%
Furthermore, at the best of our knowledge, we present for the first time an algorithm Section~\ref{sec:metrics-update} which allows to compute the most diffused link detection and link prediction metrics within a data streaming application.
