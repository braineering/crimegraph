\section{New Metrics}
\label{sec:new-metrics}

% BA

We present here \textit{Broker Allocation (BA)}, that is a local metric specifically designed for link detection in criminal networks.
 
The metric is based on the following criminological assumptions: 

\begin{enumerate}
	\item two criminal nodes hide their direct relation interposing some nodes between them, namely \textit{broker nodes}; formally, a direct link $(x,y)$ can be hidden by a path $\pi=(x,z_{1},\ldots,z_{h},y)$, where $z_{1},\ldots,z_{h}$ are broker nodes;
	
	\item the broker nodes are mainly dedicated to let them interact; formally, the total interaction weight $w_{z_{i}}=\sum_{h\in\Gamma(z_{i})}w_{z,h}$ involving a mediator node $z_{i}$ is mainly dedicated to convey the hidden interaction between $(x,y)$, that is:
	%
	\begin{equation*}
		\frac{w_{x,z_{i}}+w_{z_{i},y}}{w_{z_{i}}}=\max_{x,y\in\Gamma(z_{i})}(\frac{w_{x,z_{i}}+w_{z_{i},y}}{w_{z_{i}}})
	\end{equation*}
	
	\item the greater the number of these nodes, the greater the likelihood of hidden linkage.
\end{enumerate}

Let us now consider an undirected weighted graph $G(V,E,w)$ with $w:E\rightarrow\mathbb{R}_{\geq1}^{+}$ and a pair of nodes $x,y\in V$.
%
Under the previous assumptions, we define the metric \textit{Broker Allocation}, as follows:
%
\begin{equation}
\label{eqn:ba-local}
\Phi(x,y)=
\sum\limits_{z\in\Gamma(x)\cap\Gamma(y)}
\frac{(w_{x,z}+w_{z,y})}{w_{z}}
\end{equation}
%
where 
$\Gamma(x,y)$ is the set of common neighbors between $x$ and $y$,
$w_{x,z}$ is the weight of the edge $(x,z)$ and
$w_{z}$ is the total weight of the edges incident to $z$.

The metric differs from the traditional ones because it takes into account the distribution of the weight on the edges connecting $x$ and $y$ with their common neighbors.

% RBA

The previous metric can be extended to a version with range in $(0,1)$, namely \textit{Relative Broker Allocation (RBA)}:

\begin{equation}
\label{eqn:rba-local}
\Phi(x,y)=
\frac{\sum\limits_{z\in\Gamma(x)\cap\Gamma(y)}\frac{(w_{x,z}+w_{z,y})}{w_{z}}}
{\sum\limits_{z\in\Gamma(x)\cap\Gamma(y)}w_{z}}
\end{equation}

This metric produces a score $\Phi(x,y)\in(0,1)$, hence we can rely on a threshold to identify an hidden link. Formally, given an arbitrary threshold $\tilde{\Phi}$, every edge $(x,y)$ such that $\Phi(x,y)\geq\tilde{\Phi}$ is considered an hidden link with a probability proportional to the computed score.

Since $\Phi$ is strictly related to the edge weights, the threshold must be chosen according to the maximum and minimum weights observed on the graph.

%The Equation~\ref{eqn:nta-local} can be extended to the following quasi-local metric. Let us now consider an undirected weighted graph $G$ with $w:E\rightarrow\mathbb{R}_{\geq1}^{+}$, a pair of nodes $x,y\in V$, a locality degree $t\in \mathbb{Z}$, $t\geq 1$, and a vector $\vec{\mu}=(\mu_{1}\ldots\mu_{t})$ such that $\sum_{i=1}^{t}\mu{i}=1$.
%We can build the following quasi-local metric:

%\begin{equation}
%\label{eqn:detection-quasi-local-1}
%\Phi_{QL}(x,y,t,\vec{\mu})=\vec{\mu}\cdot\vec{\Phi}_{QL}(x,y,t)
%\end{equation}

%where $\vec{\Phi}_{QL}(x,y,t)=(\Phi_{QL}(x,y,1)\ldots\Phi_{QL}(x,y,t))$ is a vector of scores with every element defined as follows:
%\begin{equation}
%\label{eqn:detection-quasi-local-2}
%\Phi_{QL}(x,y,i)=
%\frac{\sum\limits_{\pi\in\Pi_{x,y}^{(i+1)}}w_{\pi}}
%{\sum\limits_{z\in\pi\in\Pi_{x,y}^{(i+1)}}w_{z}}
%\end{equation}

%where $\Pi_{x,y}^{(i)}$ is the set of paths between $x$ and $y$ of length $i$,
%$w_{\pi}$ is the weight of path $\pi$ and
%$w_{z}$ is the total weight of edges incident to $z$.

%The metric produces a normalized score $\Phi_{QL}(x,y,t,\vec{\mu})\in(0,1)$, hence holds here he same considerations about thresholding stated for the local metric.

% RRA

Many different techniques have been implemented for the task of link prediction, whose assumptions are the foundations for the metrics presented in Section~\ref{sec:classical-metrics}.
%
For example, the cardinality of the common neighbors set between two nodes represents an index of future interaction between them.
%
%
In this work, we consider the metric \textit{Resource Allocation (RA)} \cite{zhou2009predicting}. 
%
RA quantifies the similarity between two vertices $x$ and $y$ as the amount of unit resource sent by $x$ to $y$ through its neighbors.
%
Extensive tests in many complex networks have shown that RA is one of the best indicators on a wide variety of benchmarking datasets \cite{berlusconi2016link, Lu2011}.

In the following we present the \textit{Relative Resource Allocation (RRA)}, which extends RA by considering the $k_{z}$-th part of allocated resource unit as the index of similarity.
%
Let us consider an undirected graph $G(V,E)$ and a pair of nodes $x$,$y\in V$. We can build the following local metric:
%
\begin{equation}
\label{eqn:rra-local}
\Psi(x,y)=
\frac{\sum\limits_{z\in \Gamma(x) \cap \Gamma(y)}\frac{1}{k_{z}}}
{\sum\limits_{z\in \Gamma(x) \cap \Gamma(y)}k_{z}}
\end{equation}
%
where $k_{z}$ is the degree of node $z$ and $\Gamma(x) \cap \Gamma(y)$ is the set of common neighbors between $x$ and $y$. 
%
The role of the common neighbors $z$ in connecting $(x, y)$ is diluted if $z$ has many connections, since it will have less resources allocated to bridge $(x, y)$ \cite{berlusconi2016link}.

%Let us now consider an undirected graph $G$, a pair of nodes $x,y\in V$, a locality degree $t\in \mathbb{Z}$, $\geq 1$ and a vector $\vec{\mu}=(\mu_{1}\ldots\mu_{t})$ such that $\sum_{i=1}^{t}\mu_{i}=1$. 
%We denote with $\Theta(x,t)$ the set of nodes at distance $t$ from $x$. In another words, given a nodes $v \in G$, and a path $\pi_{x,v}$ of length $t$, $\pi_{x,v} = u,v_{2},...,v_{t-1},v$, the set $\Theta(x,t) = \{v\}$.

%Now, we can build the following quasi-local metric:

%\begin{equation}
%\label{eqn:prediction-quasi-local-1}
%\Psi_{QL}(x,y,t,\vec{\mu})=\vec{\mu}\cdot\vec{\Psi}_{QL}(x,y,t)
%\end{equation}

%where $\vec{\Psi}_{QL}(x,y,t)=(\Psi_{QL}(x,y,1)\ldots\Psi_{QL}(x,y,t))$ is a vector of scores with every element defined as follows:

%\begin{equation}
%\label{eqn:prediction-quasi-local-2}
%\Psi_{QL}(x,y,i)=
%\frac{\sum\limits_{z\in \Gamma(x,i) \cap \Gamma (y,i)}\frac{1}{k_{z}}}
%{\sum\limits_{z\in \Gamma(x,i) \cap \Gamma (y,i)}k_{z}}
%\end{equation}

%where $\Gamma(x,i) \cap \Gamma(y,i)$ is the set of common neighbors between $x$ and $y$ distant $i$ steps from each.

%The metrics in Equation~\ref{eqn:prediction-local} and  \ref{eqn:prediction-quasi-local-1}, produce a normalized score $\Psi(x,y) \in (0,1)$ and $\Psi_{QL}(x,y,t,\vec{\mu}) \in (0,1)$, respectively, so the same considerations about  threshold are still valid. 

The metric in Equation~\ref{eqn:rra-local} produces a score $\Psi (x,y) \in (0,1)$, so the same considerations on the threshold for~\eqref{eqn:rba-local} holds true for $\Psi(x,y)$, with respect to a given threshold $\tilde{\Psi}$.

