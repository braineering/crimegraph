\section{Conclusions}
\label{sec:conclusions}

The application of Big Data analytics on criminal networks has a disruptive potential for the data-driven fight against the organized crime, opening new challenges for the field of social network analysis. In this context it is necessary to develop new models for social networks analytics, ready to be deployed in a data stream environment. 

In this work we defined three new metrics for link detection and prediction in an evolving criminal network.
%
Furthermore, we proposed an algorithm to deploy both the new and the traditionally adopted metrics in a data stream environment.
%
The experiments were aimed at comparing traditional and new metrics and the proposed new ones in detecting and predicting criminal patterns, leveraging our data stream algorithm. 
%
The experimental results showed that the new metrics we proposed can reach an accuracy that is competitive with the traditional ones, both in detection and prediction.

% FUTURE WORK

The proposed models met the real-time and accuracy requirements, showing satisfying results, but can certainly be improved, as follows.
%
First of all, a stream-ready quasi-local extension of these metrics should be investigated to make them more flexible. 
%
The difficulty of this extension lies in the combinatorial explosion of metrics updates due to the bigger set of feature to examine.
%
A heuristic, possibly supported by in-memory caching features, could further improve the scalability of such an extension.

Another limitation concerns the impossibility to mine links on networks made up of more than one connected components. The native locality of the considered metrics, imposes such a limit by construction.
%
A statistical approach could overcome this limitation by allowing links mining on criminal networks containing more than one connected component. 
