\section{Architecture}
\label{sec:architecture}

A criminal network analytics application requires real-time processing of heterogeneous data from multiple distributed sources.

In an international level criminal scenario, the amount of data that must be managed reaches considerable size: a large number of people, criminals, devices, and sensors are connected via digital networks, and the cross plays among these entities generate enormous valuable information \cite{FrameworkBigdata}. For these reasons, we are interested to an high-throughput and scalable process of analysis.
Furthermore, application usage as a tool to support the law enforcement agencies, it requires the storage capacity of the social network in a database that can be consulted rapidly.

Considering these requirements, in this research we propose an application that follow a \textit{data stream processing} approach. DSP is a computational paradigm in which the sources, emit continuous data \textit{streams}, which are then processed in real time by distributed and parallelized operators. Our application uses \textit{Apache Flink} \cite{flink} as a data stream processing framework 

Data aggregation is provided by \textit{Apache Kafka} \cite{kafka}, an high-throughput message queuing system, designed for making streaming data available to consumer. Kafka implements a \textit{publish/subscribe} mechanism for \textit{asynchronous communication} of data.

Criminal network graph is stored in \textit{Neo4j} \cite{neo4j}, that is a highly scalable, native graph database. We chose Neo4j because it optimizes the storage of large graphs and provides a web-based interactive exploration of the graph.

In Figure~\ref{fig:topology} we represent the topology of operators implemented in Flink and their interactions with Kafka and Neo4j. Both link mining metrics are computed in parallel, scores are then filtered with a custom thresholding process and the output is the set of hidden links and potential links.
 
%\begin{figure}[]
%	\centering
%	\includegraphics[width=0.48\textwidth]{fig/crimegraph-layered-architecture.eps}
%	\caption{High-level architecture.}
%	\label{fig:layered-architecture}
%\end{figure}

\begin{figure*}
\centering
\includegraphics[width=6in]{./fig/topology}
\caption{The topology of architecture.}
\label{fig:topology}
\end{figure*}
