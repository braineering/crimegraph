\section{Future Work and Conclusions}
\label{sec:further-improvements}

The application of Big Data analytics on criminal networks has a disruptive potential, opening new challenges for the field of social network analysis.
%
The proposed models meet the real-time and accuracy requirements, showing satisfactory results, but can certainly be improved, as follows.

First of all, a stream-ready quasi-local extension of these metrics should be investigated to make them more flexible. 
%
The difficulty of this extension lies in the combinatorial explosion of metrics updates due to the bigger set of feature to examine.
%
A heuristic, possibly supported by in-memory caching features, could further improve the scalability of such an extension.

Another limitation concerns the impossibility to mine links on networks made up of more than one connected components. The native locality of the considered metrics, imposes such a limit by construction.
%
A statistical approach could overcome this limitation by allowing links mining on criminal networks containing more than one connected component. 

%\hlr{eliminerei questo paragrafo, non e' un improvement, ma un'altra applicazione}Furthermore, the criminological context is in line with the studies on traffic matrix estimation on IP networks  \cite{papagiannaki2004distributed}. 
%
%For instance, there could be an hidden link between two nodes, if the traffic matrix identifies a strict correlation between the outgoing and ingoing traffic from one another. 
%
%The correlation could be detected by the traffic matrix decomposition techniques \cite{elgamal2015analysis,benameur2004traffic}. 
%
The data-driven fight against the organized crime is opening new research challenges and directions. 
In this context it is necessary to develop innovative topological models, natively thought for big data analytics technologies.
In this work we proposed three new metrics for link detection and prediction, together with a data stream processing architecture able to both classical and new metrics.

The experiments are aimed at comparing classical metrics and the proposed new ones in detecting and predicting criminal patterns, leveraging our data stream algorithm. The experimental results show that the new metrics we propose can reach up to 83\% accuracy in detection, and 82\% accuracy in prediction.
