\section{Data model}
\label{sec:data-model}

A \textit{criminal network} is a social structure made up of related actors sharing some criminal intent \cite{von2001organisierte}. 
The criminological domain deals with a wide variety of entities and relations, collected by heterogeneous data sources. 
When analyzing criminal networks leveraging SNA models, this variety must be addressed \cite{pramanik2016framework}.

We model such a network with the weighted undirected graph $G(V,E,w)$, where $v\in V$ represents a \textit{criminal node} (e.g. person, group, assets or place), and $E$ represents their \textit{criminal relations}, whose typology and relevance are encapsulated by the correspondent weight.
A generic node is labeled with an IRI pointing, for example, to an external knowledge base, so as to then exploit the power of inference
A generic edge is undirected and weighted according to the Exponential Weighted Moving Average (EWMA) relevance of the observed interaction, so as to arbitrarily determining the relevance of the recent observations with respect to past ones \cite{lucas1990exponentially}. 
Such a general model makes the link mining more flexible and adaptable to different criminal contexts and investigative practices.

Furthermore, also the investigation dynamics should be considered. Since entities are observed in their interactions, isolated nodes has no sense in this model. This means that a node insertion occurs only through the insertion of an interaction involving it.

