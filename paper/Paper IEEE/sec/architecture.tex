\section{Architecture}
\label{sec:architecture}

An application for criminal network analytics requires real-time processing of heterogeneous data from multiple distributed sources.

%In an widely extended criminal scenario, the amount of data is huge, heterogeneous and generated rapidly. 
In order to extract valuable information from such datasets, the systems must be highly scalable, provide high throughput and convenient exploration of results.
%: a large number of people, criminals, devices, and sensors are connected via digital networks, and their interactions generate enormous valuable information \cite{FrameworkBigdata}. For these reasons, we are interested to an high-throughput and scalable process of analysis.
%Furthermore, application usage as a tool to support the law enforcement agencies, it requires the storage capacity of the social network in a database that can be consulted rapidly.

Considering these requirements, we propose an application that follows the \textit{data stream processing} approach, where continuous streams are processed on-the-fly by distributed stateless operators.

Our application provides an extremely flexible architecture, making it really easy to deploy and compare performances of any metric.

The data stream processing architecture is implemented with \textit{Apache Flink}\cite{flink}, a highly scalable DSP framework notable for its throughput maximization.
Data injection is provided by \textit{Apache Kafka} \cite{kafka}, a widley adopted pub/sub system natively thought to implement asychronous communication in big data analytics applications.
Criminal network graph is stored in \textit{Neo4j} \cite{neo4j}, a highly scalable graph database that optimizes the storage of large graphs and provides a convenient web-based data exploration tool.

In Figure~\ref{fig:topology} we represent the architecture of the proposed system, with a focus on the topology of stream operators. All link mining metrics are computed in parallel, scores are then filtered with a custom thresholding process and the output is the set of mined links.

\begin{figure*}
\centering
\includegraphics[width=6in]{./fig/topology}
\caption{The architecture.}
\label{fig:topology}
\end{figure*}

