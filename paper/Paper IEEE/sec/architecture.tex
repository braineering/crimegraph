\section{Architecture}
\label{sec:architecture}

An application for criminal network analytics requires real-time processing of heterogeneous data from multiple distributed sources.

%In an widely extended criminal scenario, the amount of data is huge, heterogeneous and generated rapidly. 
In order to extract valuable information from such datasets, systems must be highly scalable, provide high throughput and convenient exploration of results.
%: a large number of people, criminals, devices, and sensors are connected via digital networks, and their interactions generate enormous valuable information \cite{FrameworkBigdata}. For these reasons, we are interested to an high-throughput and scalable process of analysis.
%Furthermore, application usage as a tool to support the law enforcement agencies, it requires the storage capacity of the social network in a database that can be consulted rapidly.

Considering these requirements, we propose an application that follows a \textit{data stream processing} approach, so where continuous streams are processed on the fly by distributed stateless operators.

Our application provides an extremely flexible architecture, making it suitable to develop, deploy and compare any metric with great ease.

The data stream processing architecture is developed leveraging the \textit{Apache Flink} \cite{flink} as a DSP framework. 

Data acquisition is provided by \textit{Apache Kafka} \cite{kafka}, an high-throughput message queuing system, designed for making streaming data available to consumer. Kafka implements a \textit{publish/subscribe} mechanism for \textit{asynchronous communication} of data.

Criminal network graph is stored in \textit{Neo4j} \cite{neo4j}, that is a highly scalable, native graph database. Neo4j optimizes the storage of large graphs and provides a web-based interactive exploration tool.

In Figure~\ref{fig:topology} we represent the topology of operators implemented in Flink and their interactions with Kafka and Neo4j. Both link mining metrics are computed in parallel, scores are then filtered with a custom thresholding process and the output is the set of hidden and potential links.

\begin{figure*}
\centering
\includegraphics[width=6in]{./fig/topology}
\caption{The topology of architecture.}
\label{fig:topology}
\end{figure*}

