\section{Classical Similarity Metrics}
\label{sec:metrics}
In this section we recall the most widely adopted local metrics in link mining applications.
These metrics have been implemented to be used as an evaluation benchmark for the proposed metrics.

Given an undirected graph $G(V,E)$ and a pair of vertices $x,y\in V$, we have that:

\noindent\\
\textit{Common Neighbours (CN)}
\begin{equation}
\label{eqn:common-neighbours}
s^{CN}_{x,y}= |\Gamma(x) \cap \Gamma(y)|
\end{equation}

\noindent
\textit{Salton Index}
\begin{equation}
\label{eqn:salton}
s^{Salton}_{x,y}=
\frac{ |\Gamma(x) \cap \Gamma(y)|}{\sqrt{k_{x} \times k_{y}}}
\end{equation}

\noindent
\textit{Jaccard Index}
\begin{equation}
\label{eqn:jaccard}
s^{Jaccard}_{x,y}=
\frac{|\Gamma(x) \cap \Gamma(y)|}{|\Gamma(x) \cup \Gamma(y)|}
\end{equation}

\noindent
\textit{S\o rensen Index}
\begin{equation}
\label{eqn:sorensen}
s^{S\o rensen}_{x,y}=
\frac{2|\Gamma(x) \cap \Gamma(y)|}{k_{x} + k_{y}}
\end{equation}

\noindent
\textit{Hub Promoted Index (HPI)}
\begin{equation}
\label{eqn:hpi}
s^{HPI}_{x,y}=
\frac{|\Gamma(x) \cap \Gamma(y)|}{min\{k_{x},k_{y}\}}
\end{equation}

\noindent
\textit{Hub Depressed Index (HDI)}
\begin{equation}
\label{eqn:hdi}
s^{HDI}_{x,y}=
\frac{|\Gamma(x) \cap \Gamma(y)|}{max\{k_{x},k_{y}\}}
\end{equation}

\noindent
\textit{Leicht-Holme-Newman Index (LHN1)}
\begin{equation}
\label{eqn:lhn1}
s^{LHN1}_{x,y}=
\frac{|\Gamma(x) \cap \Gamma(y)|}{k_{x} \times k_{y}}
\end{equation}

\noindent
\textit{Preferential Attachment Index (PA)}
\begin{equation}
\label{eqn:pa}
s^{PA}_{x,y}= k_{x} \times k_{y}
\end{equation}

\noindent
\textit{Adamic-Adar Index (AA)}
\begin{equation}
\label{eqn:adamic-adar}
s^{AA}_{x,y}=
\sum\limits_{z\in \Gamma(x) \cap \Gamma(y)}\frac{1}{log(k_{z})}
\end{equation}

\noindent
\textit{Resource Allocation Index (RA)}
\begin{equation}
\label{eqn:resource-allocation}
s^{RA}_{x,y}=
\sum\limits_{z\in \Gamma(x) \cap \Gamma(y)}\frac{1}{k_{z}}
\end{equation}


where $\Gamma(x)$ is the set of neighbours of $x$ and $k_{x}$ is the degree of $x$.
