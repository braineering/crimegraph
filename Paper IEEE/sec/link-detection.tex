\section{New metrics for detection and prediction}
\label{sec:new-metrics}

The link detection is the task of discovering \textit{hidden links} in a network. 
A link $(x,y)$ is hidden if it exists in reality, but it is missing within the available datasets, because it has been neglected during investigations or it has been concealed by criminals.

We present here the \textit{Traffic Allocation (TA)} \hlr{perche' allocation? La metrica non e' una statistica sul peso del percorso tra due nodi relativo al peso del nodo attraversato?} metric, that is a local metric specifically designed for link detection in criminal networks.
 
The metric is based on the following criminological assumptions: 

\begin{enumerate}
	\item two criminal nodes hide their direct relation interposing some nodes between them, namely \textit{mediator nodes}; formally, a direct link $(x,y)$ can be hidden by a path $\pi=(x,z_{1},\ldots,z_{h},y)$, where $z_{1},\ldots,z_{h}$ are broker nodes \hlr{ambiguita tra broker e mediator};
	
	\item the broker nodes are mainly dedicated to let them interact; formally, the total interaction weight $w_{z_{i}}$ involving a mediator node $z_{i}$ is mainly dedicated to convey the hidden interaction between $(x,y)$ \hlr{indicherei esplicitamente cosa si intende per ``mainly dedicated to convey the hidden interaction'' in termini piu' formali -- quindi in funzione del peso? Inoltre in precedenza formalizzerei la funzione peso che viene usata per gli edge e per i nodi; il peso dei nodi e' indipendente da quello degli edge o e' la somma dei pesi entranti/uscenti dal nodo? Quali sono esempi di funzioni peso?};
	
	\item the greater the number of these nodes, the greater the likelihood of hidden linkage. \hlr{questi nodi devono avere proprieta'? Si parla di nodi broker nel path tra due nodi x,y?}
\end{enumerate}

Let us now consider an undirected weighted graph $G(V,E,w)$ with $w:E\rightarrow\mathbb{R}_{\geq1}^{+}$ and a pair of nodes $x,y\in V$.
Under the previous assumptions, we define the \hlr{Traffic Allocation}, a local metric, as follows:

\begin{equation}
\label{eqn:ta-local}
\Phi(x,y)=
\sum\limits_{z\in\Gamma(x)\cap\Gamma(y)}
\frac{(w_{x,z}+w_{z,y})}{w_{z}}
\end{equation}

where 
$\Gamma(x,y)$ is the set of common neighbors between $x$ and $y$,
$w_{x,z}$ is the weight of the edge $(x,z)$ and
$w_{z}$ is the total weight of the edges incident to $z$.

\hlr{Se e' nuova rispetto alla letteratura, indicherei in cosa si differenzia con quella considerata allo stato dell'arte. Se non e' nuova, riscriverei questo paragrafo indicando il riferimento in cui questa metrica e' stata definita o riconosciuta come migliore rispetto alle altre esistenti.}

The previous metric can be extended to its normalized version, namely \textit{Normalized Traffic Allocation (NTA)}:

\begin{equation}
\label{eqn:nta-local}
\Phi(x,y)=
\frac{\sum\limits_{z\in\Gamma(x)\cap\Gamma(y)}\frac{(w_{x,z}+w_{z,y})}{w_{z}}}
{\sum\limits_{z\in\Gamma(x)\cap\Gamma(y)}w_{z}}
\end{equation}

The last metric produces a normalized score $\Phi(x,y)\in(0,1)$, hence it is suitable for thresholding.
Given an arbitrary threshold $\tilde{\Phi}$, every edge $(x,y)$ such that $\Phi(x,y)\geq\tilde{\Phi}$ is considered an hidden link with a probability proportional to the score achieved.
Since $\Phi$ is strictly related to the edge weights, the threshold must be chosen according to the maximum and minimum weights observed on the graph. 

\hlr{non sono convinto questa metrica sia realmente normalizzata, perche' come denominatore c'e' $ 1 / w^2 $.}

%The Equation~\ref{eqn:nta-local} can be extended to the following quasi-local metric. Let us now consider an undirected weighted graph $G$ with $w:E\rightarrow\mathbb{R}_{\geq1}^{+}$, a pair of nodes $x,y\in V$, a locality degree $t\in \mathbb{Z}$, $t\geq 1$, and a vector $\vec{\mu}=(\mu_{1}\ldots\mu_{t})$ such that $\sum_{i=1}^{t}\mu{i}=1$.
%We can build the following quasi-local metric:

%\begin{equation}
%\label{eqn:detection-quasi-local-1}
%\Phi_{QL}(x,y,t,\vec{\mu})=\vec{\mu}\cdot\vec{\Phi}_{QL}(x,y,t)
%\end{equation}

%where $\vec{\Phi}_{QL}(x,y,t)=(\Phi_{QL}(x,y,1)\ldots\Phi_{QL}(x,y,t))$ is a vector of scores with every element defined as follows:
%\begin{equation}
%\label{eqn:detection-quasi-local-2}
%\Phi_{QL}(x,y,i)=
%\frac{\sum\limits_{\pi\in\Pi_{x,y}^{(i+1)}}w_{\pi}}
%{\sum\limits_{z\in\pi\in\Pi_{x,y}^{(i+1)}}w_{z}}
%\end{equation}

%where $\Pi_{x,y}^{(i)}$ is the set of paths between $x$ and $y$ of length $i$,
%$w_{\pi}$ is the weight of path $\pi$ and
%$w_{z}$ is the total weight of edges incident to $z$.

%The metric produces a normalized score $\Phi_{QL}(x,y,t,\vec{\mu})\in(0,1)$, hence holds here he same considerations about thresholding stated for the local metric.