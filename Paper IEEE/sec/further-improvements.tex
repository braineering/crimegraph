\section{Further improvements}
\label{sec:further-improvements}

The application of big data analytics on criminal networks has a disruptive potential, opening new challenges for the field of social network analysis.
The proposed metrics meet the critical requirements, showing satisfactory results, but can certainly be improved. 
First of all, a scalable quasi-local extension of these metrics should be investigated to make them more flexible.
Then, their main limitation concerns the impossibility to mine links on networks made up of more than one connected components. 

A statistical approach could certainly overcome some of the limitations affecting structural metrics. In fact, their main limitation concerns the impossibility to mine links on criminal networks made up of more than one connected components.  
The criminological context is in line with the studies on traffic matrix estimation on IP networks  \cite{papagiannaki2004distributed}. 
For instance, there could be an hidden link between two nodes, if the traffic matrix identifies a strict correlation between the outgoing and ingoing traffic from one another. 
The correlation could be detected by the traffic matrix decomposition techniques \cite{elgamal2015analysis}. 
In this sense, the design of efficient data stream processing algorithms for the decomposition of sparse matrices will play a key role in the criminological domain.