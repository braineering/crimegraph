\section{Introduction}
\label{sec:introduction}

The \textit{social network analysis (SNA)} is an interdisciplinary field from social sciences, statistics, graph theory and computer science. SNA has received considerable attention from the scientific community, in order to find algorithmic solutions to extract missing information, identify hidden interactions between individuals, and so on.
 
There are many domains that can be represented with a social network and to do that on network analysis, for example social interactions between people, biological interaction between proteins, systems informations. An additional field of application relates to criminal networks, to facilitate the authorities in the investigation of organized crime, such as terrorism, drug trafficking, fraud, armed gang crimes and others\cite{xu2005criminal}. 

INTRODURRE LA DEFINIZIONE DI RETE CRIMINALE.

The analysis on large volumes of data produced by a criminal network, such as phone records, bank transitions, interceptions, sales of weapons or vehicles, etc, significantly reduce the raw data and to provide the mechanisms to study the structural hidden properties of network\cite{xu2005criminal}.

DEFINIZIONE SNA

DEFINIZIONE CRIMINAL NETWORK

SNA,DSP UTILI ALLA CRIMINAL NETWORK ANALYTICS.

In Section~\ref{sec:background} we give the SNA background, useful to better understand our work.
In Section~\ref{sec:architecture} we show the reference data stream processing architecture.
In Section~\ref{sec:data-model} we define the data model adopted to describe links in criminal networks.
In Section~\ref{sec:link-detection} we introduce the task of link detection, proposing a new local metric to discover hidden links.
In Section~\ref{sec:link-prediction} we introduce the task of link prediction, proposing a new local metric to discover potential links.
In Section~\ref{sec:metrics-update} we show the algorithm responsible to detect the necessary metrics updates in a data stream processing environment.
In Section~\ref{sec:implementation} we present the technologies used to implement the application.
In Section~\ref{sec:evaluation} we give some preliminary results about the performance achieved with the proposed metrics.
In Section~\ref{sec:further-improvements} we outline the possible improvements for the proposed metrics na the general task of link detection/prediction.
In Section~\ref{sec:conclusions} we summarize our work and results.
