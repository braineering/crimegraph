\section{Background}
\label{sec:background}

We assume that the reader is familiar with the fundamentals of graph theory. 
However, we make a brief recap to introduce what we will see in the next sections.

Let $G(V,E,w)$ be an undirected weighted graph, where $V$ is the set of vertices, $E$ is the set of edges and $w:E\rightarrow\Re$ is the weight function. 
A \textit{path} $\pi_{x,y}$ between two vertices $x$,$y$ of \textit{length} $k$ in $G$ is a sequence of vertices $v_{1},v_{2},\ldots,v_{k}$, where $x = v_{1}$ and $y = v_{k}$, such that the edge $(v_{i},v_{i+1}) \in E$ for $i = 1, 2,\ldots,k-1$. 
The distance $d(x,y)$ between $x$ and $y$, is the length of the shortest path from $x$ to $y$ (or vice versa).
A graph is \textit{connected} if there is a path between every pair of vertices; otherwise, it is not connected and made of at least two \textit{connected components}.
Given two vertices $x,y \in V$, we denote with $\Gamma_{x}$ the set of neighbors of $x$ (i.e. adjacent vertices) and with $\Gamma_{x,y} = \Gamma_{x} \bigcap \Gamma_{y}$ the set of common neighbors between $x$ and $y$.

A criminal network is a social structure made up of related agents sharing some criminal intent.
We represent a criminal network with the graph $G(V,E,w)$, where $v\in V$ represents a person, a group, an assets or a place, and $E$ represents their relations, for which the tipology and relevance are encapsulated by the correspondent weight.  

% sostituito dalle due frasi precedenti
%with the topological structure of a network, in which each node $v \in V$ is an individual or an organization, and each edge $e=(u,v)\in E$ represents a link (i.e. an interaction) between two nodes, $u$ and $v$. 

% INSERIRE DEFINIZIONE DI LINK MINING, LINK DETECTION E PREDICTION

% inserirei nella introduzione, perchè non è un background, ma la spiegazione di ciò di cui ci occupiamo
%In this paper, we focus on a criminal scenario and describe it on a graph where apply the link mining techniques. In particular, we want to predict the likelihood that a link, between two entities, is hidden or that may arise in the future. These problems are called respectively the \textit{link detection problem} and \textit{link prediction problem}. 

% da inserire nella definizio di link detection e prediction (sopra)
%Briefly, we refer to the first case when the link between two nodes may already exist, but not visible, while we refer to the second case to indicate the prediction of future interaction between two observed individuals\cite{Hasan2011}.

% lo inserirei nella evaluation
%Formally, for two istants $t$ and $t' > t$, we denote $G[t,t']$ the subgraph of G consisting of all edges with a timestamp between $t$ and $t'$. Identified four times: $t_{0}, t'_{0}, t_{1}, t''_{1}$, with $t_{0}<t'_{0}<t_{1}<t'_{1}$, we refer to $[t_{0},t'_{0}]$ as the \textit{training interval} and $[t_{1},t'_{1}]$ as the \textit{test interval}. Applying a \textit{prediction algorithm} on the graph obtained after the training interval, as $G[t_{0},t'_{0}]$, we want to make a prediction on the edges that will be present in the graph $G[t_{1},t'_{1}]$ and not present in $G[t_{0},t'_{0}]$\cite{Liben-Nowell}.
%In this work, the notions of training interval and test interval, are provided with the only purpose to test the algorithm that will be presented in the following. After all, this application produce an output of links mining in real time: in which each instant $t\textsuperscript{*}>t_{0}$ identifies a graph $G[t_{0},t\textsuperscript{*}]$ on which to execute the algorithm. More details about the real time processing are explained in Section \ref{sec:architecture}.

Currently, the problem of link prediction, as the link detection, is a long-standing challenge in modern information science. 
The mainstreaming class of algorithms are the \textit{similarity-based algorithms}. %inserire citazione
Other are based on \textit{Markov chains} or \textit{statistical models}, as extensively discussed in \cite{Liben-Nowell} and \cite{Lu2011}.

%sostituito dalle due frasi precedenti.
%There are many algorithms for this problem and ways to address this challenge. 
%The mainstreaming class of algorithms so-called \textit{similarity-based algorithms}. Other algorithms are based on \textit{Markov chains} or on \textit{statistical models}, as extensively discussed in \cite{Liben-Nowell} and \cite{Lu2011}.

Let us consider the link prediction, because the link detection is analogous, the metrics used for this task are the same adopted: the difference concerns how the results are interpreted. The similarity metrics are so many and they and differ because the size in which focus the analysis. 
In detail, let $s_{x,y}$ the likelihood, i.e. the \textit{score}, as \textit{proximity} or \textit{similarity}, that between the nodes $x$,$y$ there will be a link in the future, provided by the algorithm that implements the similarity considered metric, and let $\pi_{x,y}$ a path between them, we denote \textit{local indices} the metrics that can only compute the $s_{x,y}$ at distance $d_{\pi}(x,y) = 2$ (i.e it's applicable on only nodes that have a common neighbor $\in \Gamma_{x,y}$). 

Other types of metrics are \textit{quasi-local} indices, that consider $d(x,y) \geq t$, where $t\in \mathbb{Z}, t\geq 1 $, as \textit{locality degree}, identifies a neighborhood with a max length equal to $t$ steps, starting from the node $x$ to $y$. 
Quasi-local metrics are a good tradeoff of accuracy and computational complexity, because the global indices require considerable computational effort, while local indexes are limited. 
Another metrics are \textit{global indices}, that computes the score between any pairs of nodes in $G$ not directly connected\cite{Lu2011}. The metrics used in this work are presented in Section~\ref{sec:link-detection,sec:link-prediction}.

% eliminerei questa frase: link detection non è la normalizzazione nè di RA nè di LP. Il fatto che la metrica di link prediction sia la normalizzazione di RA lo scriverei nella sezione di link prediction.
%The metrics that will present in the following Sections~\ref{sec:link-detection,sec:link-prediction}, are a normalized extension of \textit{Resource Allocation Index (RA)} and \textit{Local Path (LP)}\cite{Lu2011,zhou2009predicting,lu2009similarity}.

% leverei la descrizione di RA e LP, lasciando solamente la citazione dell'articolo che ne parla.
%, that we report for simplicity:
%\begin{equation}
%\label{eqn:resource-allocation}
%s_{xy}\textsuperscript{RA} = 
%\sum\limits_{z\in\Gamma_{x,y}}\frac{1}{k_{z}}
%\end{equation}
%where $k_{z}$ is the degree of node $z$.
%
%\begin{equation}
%\label{eqn:local-path}
%s_{xy}\textsuperscript{LP(k)} = 
%A^{2} + \epsilon A^{3} + \epsilon^{2}A^{4}+...+\epsilon^{k-2}A^{k}
%\end{equation}
%where $k\geq 1$ is a locality degree, $A$ is the adjacency matrix, and $\epsilon$ is a free parameter. For example, $A_{xy}^{2}$ is also the number of different paths with length $2$ connecting $x$ and $y$.

% The RA is an local index and LP is a quasi-local index. 
%Both metrics for link prediction assigned to a pair $x$,$y$ a score, representing their possible future relationship. This score can be viewed as a measure of similarity between nodes $x$ and $y$. Notice that for each pair of nodes, $x$, $y \in V$, we have that $s_{xy} = s_{yx}$. Performing an ranking among the RA score on a graph $G$, can be get the top most likely link those still not present in G.

% Da inserire in evaluation
%Some precision indices are usable to evaluate the accuracy of the metric applied, as we shall see in the evaluation section.

% Da inserire in evaluation
%Notice that, in this work we assume that the graph is connected, therefore each pair of nodes is connected by a path. If we had in the case of a not connected graph, with a large connected component, as \textit{giant component}, and other small size connected components, the accuracy would be distorted by different \textit{density} of edges in a connected components.

%Link mining refers to data mining techniques that consider data links when building descriptive or predictive models \cite{getoor2005link}.
