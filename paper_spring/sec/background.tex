\section{Background}
\label{sec:background}
We assume that the reader is familiar with the notation of graph theory. 
However, we make a brief recap to introduce what we will see in the next sections.

Let $G(V,E,w)$ be an undirected weighted graph, where $V$ is the set of vertices, $E$ is the set of edges and $w:E\rightarrow\Re$ is the weight function. 
A \textit{path} $\pi_{x,y}$ between two vertices $x$,$y$ of \textit{length} $k$ in $G$ is a sequence of vertices $v_{1},v_{2},\ldots,v_{k}$, where $x = v_{1}$ and $y = v_{k}$, such that the edge $(v_{i},v_{i+1}) \in E$ for $i = 1, 2,\ldots,k-1$. 
The distance $d(x,y)$ between $x$ and $y$, is the length of the shortest path from $x$ to $y$ (or vice versa).
A graph is \textit{connected} if there is a path between every pair of vertices; otherwise, it is not connected and made of at least two \textit{connected components}.
Notice that, in this work we assume a graph made up of one connected component only, namely \textit{giant component}.
Given two vertices $x,y \in V$, we denote with $\Gamma_{x}$ the set of neighbors of $x$ (i.e. adjacent vertices) and with $\Gamma_{x,y} = \Gamma_{x} \bigcap \Gamma_{y}$ the set of common neighbors between $x$ and $y$.

The \textit{link mining} refers to data mining techniques that consider data links when building descriptive or predictive models \cite{getoor2005link}. It comprises community detection, entity classification, link detection, link prediction and more. 
%In this work we focus on link detection and prediction, namely the task of determining existent though invisible links, and the prediction of future interactions\cite{Hasan2011}.
Currently, these problems are a long-standing challenge in modern information science, and the mainstream solutions leverages algorithms based on \textit{structural metrics} and \textit{statistical models} \cite{Liben-Nowell,Lu2011}.
The structural approach classifies links according to metric scores that encapsulate relevant topological features. 
These metrics can be \textit{local}, \textit{quasi-local} and \textit{global}, depending on the locality of the topological features examined with respect to the considered entities. 

When tuning the locality degree, for a specific network, the  tradeoff between accuracy and computational complexity must be addressed.

Quasi-local metrics are a good tradeoff of accuracy and computational complexity, because the global indices require considerable computational effort, while local indexes are limited. 
Typically, link detection and prediction are based on the same metrics, mainly differing in the interpretation of experimental results.

%Let us consider the link prediction, because the link detection is analogous, the metrics used for this task are the same adopted: the difference concerns how the results are interpreted. The similarity metrics are so many and they and differ because the size in which focus the analysis. 
%In detail, let $s_{x,y}$ the likelihood, i.e. the \textit{score}, as \textit{proximity} or \textit{similarity}, that between the nodes $x$,$y$ there will be a link in the future, provided by the algorithm that implements the similarity considered metric, and let $\pi_{x,y}$ a path between them, we denote \textit{local indices} the metrics that can only compute the $s_{x,y}$ at distance $d_{\pi}(x,y) = 2$ (i.e it's applicable on only nodes that have a common neighbor $\in \Gamma_{x,y}$).

%Other types of metrics are \textit{quasi-local} indices, that consider $d(x,y) \geq t$, where $t\in \mathbb{Z}, t\geq 1 $, as \textit{locality degree}, identifies a neighborhood with a max length equal to $t$ steps, starting from the node $x$ to $y$. Quasi-local metrics are a good tradeoff of accuracy and computational complexity, because the global indices require considerable computational effort, while local indexes are limited. 

%Other types of metrics are \textit{quasi-local} indices, that consider $d(x,y) \geq t$, where $t\in \mathbb{Z}, t\geq 1 $, as \textit{locality degree}, identifies a neighborhood with a max length equal to $t$ steps, starting from the node $x$ to $y$.
%Another metrics are \textit{global indices}, that computes the score between any pairs of nodes in $G$ not directly connected\cite{Lu2011}. The metrics used in this work are presented in Section~\ref{sec:link-detection,sec:link-prediction}.

% leverei la descrizione di RA e LP, lasciando solamente la citazione dell'articolo che ne parla.
%, that we report for simplicity:
%\begin{equation}
%\label{eqn:resource-allocation}
%s_{xy}\textsuperscript{RA} = 
%\sum\limits_{z\in\Gamma_{x,y}}\frac{1}{k_{z}}
%\end{equation}
%where $k_{z}$ is the degree of node $z$.
%
%\begin{equation}
%\label{eqn:local-path}
%s_{xy}\textsuperscript{LP(k)} = 
%A^{2} + \epsilon A^{3} + \epsilon^{2}A^{4}+...+\epsilon^{k-2}A^{k}
%\end{equation}
%where $k\geq 1$ is a locality degree, $A$ is the adjacency matrix, and $\epsilon$ is a free parameter. For example, $A_{xy}^{2}$ is also the number of different paths with length $2$ connecting $x$ and $y$.

%Both metrics for link prediction assigned to a pair $x$,$y$ a score, representing their possible future relationship. This score can be viewed as a measure of similarity between nodes $x$ and $y$. Notice that for each pair of nodes, $x$, $y \in V$, we have that $s_{xy} = s_{yx}$. Performing an ranking among the RA score on a graph $G$, can be get the top most likely link those still not present in G.


